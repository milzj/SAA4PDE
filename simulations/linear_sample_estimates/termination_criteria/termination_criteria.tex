\documentclass{scrartcl}

\usepackage[hidelinks]{hyperref}
\usepackage{ifthen}
\usepackage{amsmath}


\newcommand{\prox}[3][]{\mathrm{prox}_{#2}(#3)}
\newcommand{\norm}[2][2]{\|#2\|_{#1}}

\title{Termination criteria}
\date{April 15, 2022}

\begin{document}

\maketitle

We relate the ``termination criterion'' 
$$\norm[U]{\alpha v + F(\prox{\varphi/\alpha}{v})} \leq \varepsilon$$
to 
$$\norm[U]{u-\prox{\varphi/\alpha}{-(1/\alpha)F(u)}} \leq \delta,$$
where $u = \prox{\varphi/\alpha}{v}$.
More precisely, given the validity of the first estimate, we derive
an upper bound on $\delta$ such that the second inequality holds true.

The evaluation of the first inequality requires the point $v$. The second one
can be evaluated using $u = \prox{\varphi/\alpha}{v}$.
The mapping  $v \mapsto \alpha v+F(\prox{\varphi/\alpha}{v})$ 
defines a normal map.


Let $U$ be a real Hilbert space 
and let $v \in U$. Moreover let $\varphi : U \to (-\infty,\infty]$ 
be proper, closed, and convex, let
$\alpha > 0$, and let $F : U \to U$ be a mapping.
We define $u = \prox{\varphi/\alpha}{v}$.
Here $\prox{\varphi/\alpha}{\cdot}$ is the proximity operator of
the function $\varphi/\alpha$.

Suppose that $\norm[U]{\alpha v + F(\prox{\varphi/\alpha}{v})} \leq \varepsilon$
for some $\varepsilon \geq 0$. Since $\prox{\varphi/\alpha}{\cdot}$ is
firmly nonexpansive, we have


\begin{align*}
	\norm[U]{u-\prox{\varphi/\alpha}{-(1/\alpha)F(u)}}
	& = 
	\norm[U]{\prox{\varphi/\alpha}{v}-\prox{\varphi/\alpha}{-(1/\alpha)F(u)}}
	\\
	& \leq
	\norm[U]{v+(1/\alpha)F(u)}
	\\
	& = (1/\alpha)\norm[U]{\alpha v+ F(u)}
	\\
	& = (1/\alpha)\norm[U]{\alpha v+ F(\prox{\varphi/\alpha}{v})}
	\\
	& \leq \varepsilon/\alpha.
\end{align*}

If $\varphi = 0$, then the first estimate is an identity.


\end{document}
